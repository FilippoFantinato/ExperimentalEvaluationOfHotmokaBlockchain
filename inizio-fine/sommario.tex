% !TEX encoding = UTF-8
% !TEX TS-program = pdflatex
% !TEX root = ../tesi.tex

%**************************************************************
% Sommario
%**************************************************************
\cleardoublepage
\phantomsection
\pdfbookmark{Sommario}{Sommario}
\begingroup
\let\clearpage\relax
\let\cleardoublepage\relax
\let\cleardoublepage\relax

\chapter*{Sommario}

Il presente documento descrive il tirocinio da me svolto presso l'azienda Sync Lab s.r.l durante il periodo che va dal 03-05-2021 al 25-06-2021.
L'esperienza di tirocinio ha avuto una durata complessiva di 320 ore ed è stata supervisionata e coordinata sia dal mio tutor aziendale, l'ingegnere Fabio Pallaro, sia dal mio relatore presso l'ateneo, \profTitle{} \myProf. \\

\noindent Lo scopo dello stage era di realizzare uno \textit{smart contract} che gestisse la compravendita di NFT. Questo \textit{smart contract} doveva essere implementato attraverso gli standard NFT delle \textit{blockchain} Ethereum e Hotmoka. Inoltre ne è seguito anche lo sviluppo di una libreria che ha permesso alla piattaforma NFTLab di comunicare con la blockchain.

\noindent Il percorso di tirocinio ha richiesto lo studio della tecnologia \textit{blockchain}, approfondendo Ethereum e Hotmoka, e l'implementazione di \textit{smart contract} attraverso i linguaggi di programmazione Solidity e Takamaka seguendo lo standard ERC721. \\

\noindent Ho suddiviso il presente documento in 4 capitoli:
\begin{itemize}
  \item \hyperref[cap:contesto-aziendale]{\textbf{Capitolo 1}}: presentazione del contesto organizzativo e produttivo aziendale, con approfondimento sui processi aziendali interni e sulla propensione all' innovazione;

  \item \hyperref[cap:stage]{\textbf{Capitolo 2}}: presentazione dell'offerta di stage con gli obiettivi, i vincoli e le motivazioni che mi hanno spinto a scegliere l'azienda Sync Lab; 
  
  \item \hyperref[cap:nftlab]{\textbf{Capitolo 3}}: presentazione dettagliata del progetto con le relative fasi, introduzione delle tecnologie trattate e delle soluzioni progettuali attuate per la realizzazione dei prodotti attesi dall'azienda;
  
  \item \hyperref[cap:valutazione-finale]{\textbf{Capitolo 4}}: valutazione retrospettiva del percorso di tirocinio, riguardante gli obiettivi raggiunti, le difficoltà e le competenze professionali maturate;
  
  \item \textbf{Glossario}: al suo interno sono stati riportati tutti i termini ambigui contrassegnati da una \textbf{(g)} al pedice;
  
  \item \textbf{Acronimi e Abbreviazioni}: al suo interno sono stati riportati tutti gli acronimi e abbreviazioni;

  \item \hyperref[cap:bibliografia-sitografia]{\textbf{Bibliografia e Sitografia}}: al suo interno sono stati riportati tutti i libri e siti dai quali ho reperito le informazioni.
\end{itemize}

%\vfill
%
%\selectlanguage{english}
%\pdfbookmark{Abstract}{Abstract}
%\chapter*{Abstract}
%
%\selectlanguage{italian}

\endgroup

\vfill

