% !TEX encoding = UTF-8
% !TEX TS-program = pdflatex
% !TEX root = ../../tesi.tex

\section{Differenza tra il mondo universitario e lavorativo}
Durante il mio tirocinio curricolare presso l'azienda Sync Lab ho potuto analizzare un vero ambiente lavorativo e farmi un'idea ben chiara della differenza rispetto a quello universitario. Il divario tra università e lavoro risulta ampio per quanto riguarda la metodologia di lavoro, i progetti e le tecnologie utilizzate. Tutti i progetti svolti durante il mio percorso universitario sono ai fini del corso con requisiti ben chiari ed avendo una data di consegna del \textit{software} che definisce anche la sua morte. Questi aspetti non ci permettono di sperimentare le incomprensioni che si generano durante le varie fasi di sviluppo tra gli attori coinvolti e di provare la fase di manutenzione del prodotto, ovvero quella dove risulta molto importante l'aver progettato e sviluppato un \textit{software} di qualità. Inoltre, i progetti universitari spesso ignorano aspetti essenziali nel mondo reale, uno tra questi la sicurezza.
I progetti nel mondo del lavoro sono molto diversi, infatti quasi sempre non si hanno requisiti ben definiti, da trovare con il proponente, e la metodologia di lavoro è molto più frenetica in quanto il cliente vuole un prodotto che riesca a soddisfare tutte le sue aspettative e sviluppato nel più breve tempo possibile. 
Il mondo del lavoro ha bisogno di persone con una conoscenza trasversale su tecnologie che riguardano molti ambiti, mentre invece l'università si pone come ambiente di istruzione dove apprendere i principi dell'informatica con sistematicità ed ordine. 
Questo però non bisogna vederlo come un lato negativo, ma come un dato di fatto. L'università è giusto che sia un mondo dove la teoria fa da padrone, perché solamente attraverso lo studio di come funzionano i concetti che stanno alla base dell'informatica si può davvero comprendere come funziona tutto quello che c'è costruito sopra, mentre il lavoro è l'applicazione di quanto imparato. Ritengo che siano due mondi complementari che devono coesistere e continuare ad alimentarsi insieme. \\

Posso affermare che le conoscenze apprese durante il mio percorso di studi sono state una base solida per svolgere lo stage. A livello tecnico le basi fornitemi dei linguaggio C++ e Java sono state sufficienti per studiare i linguaggi richiesti e sviluppare tutti i prodotti. Per quanto riguarda la metodologia di lavoro ha influito il progetto che ho svolto a "Ingegneria del \textit{Software}", poiché mi ha insegnato a lavorare in gruppo, una componente essenziale nel mondo del lavoro. Anche negli aspetti dell'organizzazione del progetto l'università è riuscita a darmi delle basi teoriche e pratiche che hanno facilitato di molto lo sviluppo del \textit{software}. Le uniche lacune che ho trovato nel corso di studi sono il fatto che molti corsi essenziali siano stati inseriti tra quelli opzionali. Uno tra questi è "Tecnologie \textit{Open Source}", dove ho appreso molte conoscenze riguardanti la gestione del progetto che sono risultate molto importanti durante lo svolgimento dello stage. Oltre a questo corso è presente anche "Altri Paradigmi di Programmazione", il quale mi ha formato sotto i punti di vista teorici e pratici alla base dei sistemi distribuiti. \\

In conclusione, mi ritengo soddisfatto del corso di studi appena terminato e dell'esperienza di stage effettuata, poiché entrambe mi hanno cresciuto sia a livello personale e sia a livello professionale in maniera approfondita sotto l'aspetto teorico e pratico.
