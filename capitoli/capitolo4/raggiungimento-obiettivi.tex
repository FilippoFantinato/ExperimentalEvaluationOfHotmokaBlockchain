% !TEX encoding = UTF-8
% !TEX TS-program = pdflatex
% !TEX root = ../../tesi.tex

\section{Raggiungimento degli obiettivi}
A conclusione dello stage di 320 ore ho provveduto insieme al mio \textit{tutor} aziendale, Fabio Pallaro, ad analizzare il soddisfacimento degli obiettivi dello stage e personali.

\paragraph{Obiettivi dello stage}
Per quanto riguarda gli obiettivi dello stage, posso affermare che sono stati coperti totalmente e gli \textit{stakeholders} sono rimasti molto soddisfatti di quanto è stato sviluppato. 
Sono riuscito a raggiungere il soddisfacimento totale degli obiettivi anche grazie all'analisi preventiva dei rischi che ho svolto, la quale mi ha permesso di identificare le parti critiche del progetto alle quali prestare molta più attenzione. 
Come si può vedere dal resoconto dell'analisi dei rischi (tabella \ref{tab:resoconto-analisi-dei-rischi}) non si è verificato alcun rischio preventivato. 
Ciò che mi ha rallentato, ma non è stato preventivato, era la non presenza di uno standard per la gestione di NFT in Hotmoka che è stata data per scontata. Comunque non è stato un ostacolo insormontabile e, anzi, mi ha permesso di approfondire ancora meglio il linguaggio di programmazione per gli \textit{smart contract} Takamaka.

\begin{longtabu}{|X[c,m]|X[c,m]|X[5,l,m]|}

  \hline

  \textbf{Rischio} & \textbf{Verificato} & \textbf{Resoconto} \\
  
  \hline

  RIS1 & \xmark & Non ho avuto alcuna difficoltà a comprendere gli argomenti trattati durante il periodo di studio grazie al materiale completo e ampio che il \textit{tutor} aziendale mi ha fornito \\

  \hline

  RIR2 & \xmark & Ho compreso correttamente tutti requisiti da implementare anche grazie alla metodologia Scrum che mi ha permesso di avere un costante riscontro di quanto stavo facendo \\

  \hline

  RIT3 & \xmark & Solidity è risultato un linguaggio molto semplice da imparare grazie alla sua somiglianza ad altri linguaggi come Java e C++ \\

  \hline

  RIT4 & \xmark & Non ho avuto difficoltà con le nuove versioni di Java e, anzi, ho fatto un grande uso delle nuove funzionalità \\

  \hline

  RIT5 & \xmark & Grazie a strumenti come HardHat e Ganache non ho avuto alcun problema nell'utilizzare un ambiente di \textit{test} per la \textit{blockchain} Ethereum \\

  \hline

  RIT6 & \xmark & Grazie alla documentazione molto ampia e ben strutturata di Hotmoka non ho avuto alcun problema nell'utilizzare un ambiente di \textit{test} \\

  \hline

  \caption{Resoconto dell'analisi dei rischi}
\end{longtabu}
\label{tab:resoconto-analisi-dei-rischi}

Di seguito è riportata una tabella riassuntiva degli obiettivi dello stage:

\begin{longtabu}{|X|c|}

  \hline

  \textbf{Obiettivo} & \textbf{Soddisfatto} \\

  \hline

  Documento tecnico & \cmark \\

  \hline

  Implementazione dello \textit{smart contract} per Ethereum & \cmark \\

  \hline

  Implementazione dello \textit{smart contract} per Hotmoka & \cmark \\

  \hline

  Implementazione della libreria per l'interazione del \textit{back-end} con entrambe le \textit{blockchain} & \cmark \\

  \hline

  \caption{Resoconto del soddisfacimento degli obiettivi dello stage}
\end{longtabu}

In più, come già illustrato nel \hyperref[cap:nftlab]{capitolo 3}, ho ottenuto ottimi risultati per quanto riguarda il \textit{code coverage} raggiunto per ogni prodotto sviluppato.

\begin{longtabu}{|X[5,l]|X[1,c]|X[1,c]|}
  
  \hline

  \multicolumn{1}{|c|}{\textbf{Prodotto}} & \textbf{Numero di test} & \textbf{Code coverage} \\

  \hline
  \textbf{\textit{Smart contract} per Ethereum} & \textbf{19} & \textbf{100\%} \\
  % \hline

  Test di unità & 19 & \\
  Test di integrazione & 0 & \\
  Test di sistema & 0 & \\

  \hline
  \textbf{\textit{Smart contract} per Hotmoka} & \textbf{19} & \textbf{100\%} \\
  % \hline

  Test di unità & 19 & \\
  Test di integrazione & 0 & \\
  Test di sistema & 0 & \\
  
  \hline
  \textbf{Libreria per l'interazione con gli \textit{smart contract}} & \textbf{48} & \textbf{90\%} \\
  % \hline
  
  Test di unità & 29 & \\
  Test di integrazione & 14 & \\
  Test di sistema & 5 & \\

  \hline
  
  \caption{Resoconto dei test scritti}
\end{longtabu}

\paragraph{Obiettivi personali}
Per quanto concerne gli obiettivi personali ho soddisfatto pienamente tutti gli obiettivi prefissati, riuscendo anche a fare molto di più utilizzando il protocollo IPFS per il salvataggio delle opere e sviluppando una mia implementazione dello standard ERC721 per Hotmoka. Nel complesso, perciò, gli \textit{stakeholders} sono rimasti molto soddisfatti del lavoro compiuto.

\begin{longtabu}{|c|c|}
  \hline
  \textbf{Obiettivo} & \textbf{Soddisfatto} \\
  \hline

  \personalGoalCode{O}{\ref{personal-goal:O.smart-contract}} & \cmark \\

  \hline

  \personalGoalCode{O}{\ref{personal-goal:O.library}} & \cmark \\

  \hline

  \personalGoalCode{D}{\ref{personal-goal:D.smart-contract}} & \cmark \\

  \hline

  \personalGoalCode{F}{\ref{personal-goal:F.tests}} & \cmark \\

  \hline

  \caption{Resoconto del soddisfacimento degli obiettivi personali}
\end{longtabu}
