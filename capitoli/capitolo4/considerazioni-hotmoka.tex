% !TEX encoding = UTF-8
% !TEX TS-program = pdflatex
% !TEX root = ../../tesi.tex

\section{Considerazioni sulla blockchain HotMoka}
Durante il mio stage ho potuto svolgere un progetto che mi ha permesso di sperimentare la \textit{blockchain} HotMoka ed il suo linguaggio per la scrittura di \textit{smart contract} chiamato Takamaka. Di seguito analizzerò proprio l'aspetto di scrittura dei contratti confrontandola con quella di Ethereum, una \textit{blockchain} famosa proprio per questo particolare. 
HotMoka, prima di tutto, è una \textit{blockchain} che è nata da pochi anni e tutt'ora non è ancora stata utilizzata per progetti grandi. A parer mio ha molti aspetti positivi, tra cui è presente l'utilizzo del \textit{Proof of Stake} come algoritmo del consenso rispetto al classico \textit{Proof of Work}, il quale è più lento, richiede commissioni più alte e necessita di una spesa eccessiva riguardante la corrente elettrica. \\

Per quanto riguarda la scrittura di \textit{smart contract} ritengo che attraverso il linguaggio Takamaka si riesca a sviluppare contratti più facilmente rispetto a quello che si riesce a fare con Solidity. Solidity, infatti, è un linguaggio fin troppo semplice e specifico per la scrittura di contratti. Questo a parer mio è un grande limite perché si risente la mancanza di costrutti e di tipi di dato più complessi, come le strutture dati che permettono una gestione migliore delle informazioni. Un altro aspetto negativo è la sicurezza, visto che in Solidity è molto semplice introdurre \textit{bug} e scrivere codice che sia vulnerabile a determinati tipi di attacco, se non si conosce bene il funzionamento del linguaggio.
Di fatto ho trovato Takamaka un linguaggio migliore e più adeguato per quanto riguarda la scrittura degli \textit{smart contract} anche se non ottimo, in quanto non soffre dei problemi elencati in precedenza ma ne presenta di altri, come l'esistenza del valore \textit{null} che Solidity non ha. Quello che auspico è che gli sviluppatori di HotMoka non si fermino solo al portare Java come linguaggio per la scrittura di \textit{smart contract} ma, essendo che HotMoka funziona attraverso la JVM, spero che riescano a portare linguaggi molto più avanzati come Kotlin e Scala. \\

Quello che Solidity ha in più di Takamaka è il fatto di avere una \textit{community} molto più ampia, la quale ha prodotto librerie e strumenti che facilitano lo sviluppo, elementi quasi assenti in Takamaka. Il difetto più grande di HotMoka è il non avere attualmente uno strumento che automatizzi l'integrazione di uno \textit{smart contract} con un qualsiasi altro applicativo, come Web3 con Solidity. Questa mancanza mi ha rallentato molto durante lo sviluppo del contratto in HotMoka, fatto che non è avvenuto per quanto riguarda Ethereum con Solidity. \\

In conclusione, mi sento di dire che è troppo presto per dare un giudizio sulla \textit{blockchain} HotMoka. Ritengo che abbia molte potenzialità e che riesca a dare un valore aggiunto a questo mondo, ma è ancora troppo giovane per poter essere valutata a pieno. 
