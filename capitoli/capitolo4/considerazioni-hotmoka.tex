% !TEX encoding = UTF-8
% !TEX TS-program = pdflatex
% !TEX root = ../../tesi.tex

\section{Considerazioni sulla blockchain Hotmoka}
Durante il mio stage ho potuto svolgere un progetto che mi ha permesso di sperimentare la \textit{blockchain} Hotmoka ed il suo linguaggio per la scrittura di \textit{smart contract} chiamato Takamaka. Di seguito analizzerò proprio l'aspetto di scrittura dei contratti confrontandola con quella di Ethereum, una \textit{blockchain} famosa proprio per questo particolare. 
Hotmoka, prima di tutto, è una \textit{blockchain} che è nata da pochi anni e tutt'ora non è ancora stata utilizzata per progetti reali. A parer mio ha molti aspetti positivi, uno tra questi è l'utilizzo del \textit{Proof of Stake} come algoritmo del consenso rispetto al classico \textit{Proof of Work}, il quale è più lento, richiede commissioni più alte e si ha una spessa eccessiva riguardante la corrente. \\

Per quanto riguarda la scrittura di \textit{smart contract} ritengo che attraverso il linguaggio Takamaka si riesca a sviluppare contratti più facilmente rispetto a quello che si riesce a fare con Solidity. Solidity, infatti, è un linguaggio fin troppo semplice e specifico per la scrittura di contratti. Questo a parer mio è un grande limite perché si risente la mancanza di costrutti e tipi di dato più complessi, come le strutture dati che permettono una gestione migliore delle informazioni. Un altro aspetto è la sicurezza, visto che in Solidity è molto semplice introdurre \textit{bug} e rendere facili determinati tipi di attacco se non si conosce bene il funzionamento del linguaggio.
Di fatto ho trovato Takamaka un linguaggio migliore e più adeguato per quanto riguarda la scrittura degli \textit{smart contract}, anche se non ottimo. Java ha comunque vari problemi come l'esistenza \textit{null} che Solidity non ha. Quello che auspico è che non si fermino solo al portare Java come linguaggio per la scrittura di \textit{smart contract} ma essendo che Hotmoka funziona attraverso la JVM, spero che riescano a portare linguaggi molto più avanzati come Kotlin e Scala. \\

Quello che Solidity ha in più di Takamaka è il fatto di avere una \textit{community} molto più ampia, la quale ha prodotto librerie e strumenti che ne facilitano lo sviluppo. Il difetto più grande di Hotmoka è il non avere attualmente uno strumento che automatizzi l'integrazione di uno \textit{smart contract} con un qualsiasi altro applicativo, come Web3 con Solidity. Questa mancanza mi ha rallentato molto durante lo sviluppo del contratto in Hotmoka, fatto che non è avvenuto per quanto riguarda Ethereum con Solidity. \\

In conclusione, mi sento di dire che è troppo presto per dare un giudizio sulla \textit{blockchain Hotmoka}. Ritengo che abbia molte potenzialità e che riesca a dare un valore aggiunto a questo mondo, ma è ancora troppo giovane per poter essere valutata a pieno. 
