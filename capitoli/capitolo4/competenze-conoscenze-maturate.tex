% !TEX encoding = UTF-8
% !TEX TS-program = pdflatex
% !TEX root = ../../tesi.tex

\section{Competenze e conoscenze maturate}
Durante lo svolgimento dello stage ho approfondito molti argomenti che riguardano la tecnologia \textit{blockchain} e lo sviluppo di \textit{smart contract}. Inoltre ho studiato, anche se in maniera superficiale, le reti distribuite con i relativi problemi di cui sono affette.
Oltre a questo, l'esperienza di stage è stata utile anche per quanto riguarda la crescita personale, infatti ho potuto analizzare e provare in prima persona il metodo lavorativo di un'azienda grande e strutturata come lo è Sync Lab. Il progetto di stage è stato svolto in un gruppo di 4 persone e questo mi ha permesso di migliorare il mio modo di lavorare in gruppo. 
Di seguito vengono riportate le principali conoscenze a livello professionale che ho maturato con questo tirocinio. 

\paragraph{Approfondimento della metodologia Scrum}
Durante lo stage ho potuto apprendere ed approfondire la metodologia Scrum. Questa l'avevo già studiata in precedenza ma ora l'ho sperimentata anche a livello pratico e ne ho valutato personalmente l'efficacia. Grazie a questo ho imparato ad organizzare gli obiettivi in maniera efficiente ed efficace, al fine di sviluppare un progetto in \textit{team}.

\paragraph{Conoscenze riguardanti la tecnologia blockchain} 
Grazie a questo stage ho potuto approfondire una tecnologia che mi ha sempre incuriosito, ovvero la \textit{blockchain}. Ho compreso le caratteristiche che deve avere un \textit{distributed ledger} per essere definito \textit{blockchain} ed il suo funzionamento comune a tutte le varie implementazioni. 
Inizialmente ho cominciato con lo studio di BitCoin, grazie al quale ho potuto definire un termine di paragone per le altre \textit{blockchain} studiate in seguito. Di BitCoin ho potuto apprendere il suo funzionamento in maniera approfondita, a partire da cos'è un \textit{wallet} fino al ciclo di vita di una transazione. In seguito ho eseguito lo stesso di tipo di studio sulle \textit{blockchain} Ethereum e Hotmoka, analizzando le differenze che le contraddistinguono sia tra di loro e sia da BitCoin. \\

\noindent In relazione con ciò, ho studiato e accresciuto le mie conoscenze riguardanti: 
\begin{itemize}
  \item i sistemi crittografici di firma digitale e le funzioni di hash impiegate in tutte le \textit{blockchain};
  \item le strutture dati utilizzate in BitCoin, Ethereum e Hotmoka, come il \textit{Merkle Tree} e il \textit{Merkle PATRICIA tries}, per la gestione dello stato e delle transazioni;
  \item i vari algoritmi del consenso del tipo \textit{Proof of X}, come \textit{Proof of Work}, \textit{Proof of Stake}, \textit{Delegated Proof of Stake}, \textit{Proof of Space} e \textit{Proof of Authority}, con i relativi vantaggi e svantaggi di ognuno.
\end{itemize}

Oltre a tutti questi aspetti tecnici ho incluso anche, per interesse personale, le possibili applicazioni della tecnologia \textit{blockchain} e la sua valenza legale.

\paragraph{Conoscenze riguardanti gli smart contract e NFT}
Attraverso il progetto sviluppato durante lo svolgimento dello stage ho potuto approfondire cosa sono e le caratteristiche degli \textit{smart contract}. Inoltre, ho acquisito anche conoscenze riguardanti il mondo degli NFT, cosa sono, le loro caratteristiche e le possibili applicazioni, e lo standard che regola il loro sviluppo, ovvero ERC721. Ho potuto sperimentare a livello pratico anche il loro sviluppo nelle \textit{blockchain} Ethereum e Hotmoka.

\paragraph{Nuovi linguaggi di programmazione}
Per quanto riguarda i linguaggi di programmazione ho appreso in maniera approfondita Solidity e Takamaka, i quali servono per la scrittura di \textit{smart contract} in, rispettivamente, Ethereum e Hotmoka. Inoltre ho potuto anche utilizzare le ultime versioni di Java.
In tutti i linguaggi di programmazione utilizzati ho anche approfondito la parte di analisi statica del \textit{software}, con \textit{Checkstyle}, \textit{TSLint} e \textit{Solhint}, e di scrittura dei \textit{test} automatici, con Mocha, Chai e JUnit5.

\paragraph{Approfondimento degli strumenti di gestione del progetto}
Ho appreso pienamente il funzionamento di GitHub e di tutti gli strumenti di gestione del progetto che mette a disposizione, includendoli nel mio \textit{Way Of Working}. In particolare ho potuto utilizzare lo strumento di \textit{Issue Tracking System}, le funzionalità di \textit{Continuous Integration} attraverso GitHub \textit{Action} e l'\textit{artifact repository} messa a disposizione.
