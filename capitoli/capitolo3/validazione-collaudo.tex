% !TEX encoding = UTF-8
% !TEX TS-program = pdflatex
% !TEX root = ../../tesi.tex

\section{Validazione e collaudo}
Nella fase di validazione e collaudo ho controlla le funzionalità implementate in seguito alla fase di sviluppo, analizzando approfonditamente tutti i requisiti che avevo precedentemente individuato. Per ognuno di questi requisiti ho analizzato l'effettivo soddisfacimento.

\begin{longtabu}{|c|c|}

  \hline

  \textbf{Codice requisito} & \textbf{Soddistatto} \\

  \hline

  \rfuncode{\ref{rfun:upload-new-nft}}{O}{} & \checkmark \\
  \rfuncode{\ref{rfun:transfer-nft}}{O}{} & \checkmark \\
  \rfuncode{\ref{rfun:get-nft-by-id}}{O}{} & \checkmark \\
  \rfuncode{\ref{rfun:get-nft-by-hash}}{O}{} & \checkmark \\
  \rfuncode{\ref{rfun:get-transaction-history}}{O}{} & \checkmark \\

  \hline

  \rquacode{\ref{rqua:mit}}{O} & \checkmark \\
  \rquacode{\ref{rqua:statements-coverage}}{O} & \checkmark \\
  \rquacode{\ref{rqua:branches-coverage}}{O} & \checkmark \\
  \rquacode{\ref{rqua:functions-coverage}}{O} & \checkmark \\
  \rquacode{\ref{rqua:lines-coverage}}{O} & \checkmark \\

  \hline

  \rvincode{\ref{rvin:solidity}}{O} & \checkmark \\
  \rvincode{\ref{rvin:java-version}}{O} & \checkmark \\
  \rvincode{\ref{rvin:ipfs}}{O} & \checkmark \\

  \hline

  \textbf{Percentuale requisiti soddisfatti} & 100\% \\

  \hline

\caption{Riassunto dei requisiti soddisfatti}
\end{longtabu}

Inoltre, durante questo periodo, ho stilato anche il \textbf{documento tecnico} richiesto dall'azienda. Di fatto, non è stato stilato un unico documento, come Inizialmente richiesto, ma bensì 4 documenti, ognuno per ogni \textit{repository}. \\

\noindent Ogni documento contiene al suo interno le seguenti sezioni:
\begin{itemize}
  \item \textbf{Introduzione}: dove viene introdotta la componente del progetto che è presente nella \textit{repository};
  \item \textbf{Strumenti utilizzati}: dove vengono spiegati tutti i \textit{framework}, librerie e strumenti di supporto che sono stati utilizzati;
  \item \textbf{Organizzazione della repository}: dove viene spiegata dettagliatamente il ruolo e lo scopo delle cartelle e \textit{file} più importanti;
  \item \textbf{Test sviluppati}: in cui vengono spiegati i test che sono stati scritti ed eventualmente qualche loro particolarità;
  \item \textbf{I comandi più importanti}: dove vengono spiegati i comandi per eseguire il progetto;
  \item \textbf{\textit{Continuous Integration}}: in cui sono stati spiegati i vari processi di \textit{Continuous Integration} che sono stati implementati, come ad esempio le \textit{GitHub action} che sono state scritte.
\end{itemize}

In seguito ad un'approfondita analisi eseguita anche con l'aiuto del mio \textit{tutor} aziendale, Fabio Pallaro, è stato deciso di utilizzare come implementazione da integrare all'interno del prodotto finale quella realizzata con Ethereum.
