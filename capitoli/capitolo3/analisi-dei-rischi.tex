% !TEX encoding = UTF-8
% !TEX TS-program = pdflatex
% !TEX root = ../../tesi.tex

\section{Analisi dei rischi}
Durante la fase di studio preliminare, ho proceduto all'identificazione dei rischi che potrebbero accadere durante la fase di sviluppo e, per ogni rischio identificato verrà fornita una descrizione, la probabilità che possa verificarsi e la sua Pericolosità. \\

\noindent La classificazione dei rischi ha seguito la seguente convenzione:
\begin{center}
  RI(S|R|T)[0-9]+
\end{center}
dove:
\begin{itemize}
  \item \textbf{S}: è la lettera con la quale si fa riferimento ad un rischio che si potrebbe incontrare durante il periodo di studio;
  \item \textbf{R}: è la lettera con la quale si fa riferimento ad un rischio che potrebbe essere causato dai requisiti;
  \item \textbf{T}: è la lettera con la quale si fa riferimento ad un rischio che potrebbe essere causato dalle tecnologie.
\end{itemize}

\noindent La pericolosità di ogni rischio è stata suddivisa in tre categorie:
\begin{itemize}
  \item \textbf{Impatto alto}: provoca uno sforamento significativo sulle ore previste ed è difficilmente risolvibile nel tempo di progetto;
  \item \textbf{Impatto medio}: provoca uno sforamento di qualche ora ed è risolvibile;
  \item \textbf{Impatto basso}: provoca uno sforamento trascurabile delle ore pianificate ed è facilmente risolvibile.
\end{itemize}

\begin{longtabu}{|X[1,c]|X[2,c]|X[0.8,c,m]|X[0.8,c,m]|}

  \hline

  \textbf{Codice identificativo} & \textbf{Descrizione} & \textbf{Probabilità} & \textbf{Pericolosità} \\

  \hline

  \risk{S} & Non comprensione di un argomento presentato durante il periodo di studio & Media & Media \\

  \hline

  \risk{R} & & &  \\

  \hline

  \risk{T} & Difficoltà nell'imparare il linguaggio Solidity & Media & Alta \\
  % \risk{T} & Difficoltà nell'imparare il linguaggio Solidity & &  \\

  \hline

  \caption{Analisi dei rischi}
\end{longtabu}
