% !TEX encoding = UTF-8
% !TEX TS-program = pdflatex
% !TEX root = ../../tesi.tex

\section{Analisi dei rischi}
Prima di continuare con qualsiasi altro tipo di attività riguardante il mio stage, ho proceduto all'identificazione dei rischi che potrebbero accadere durante la fase di studio e di sviluppo. 
Per ogni rischio identificato verrà fornita una descrizione, la probabilità con la quale possa verificarsi e la sua pericolosità. \\

\noindent La classificazione dei rischi ha seguito la seguente convenzione:
\begin{center}
  RI(S|R|T)[0-9]+
\end{center}
dove:
\begin{itemize}
  \item \textbf{S}: è la lettera con la quale si fa riferimento ad un rischio che si potrebbe incontrare durante il periodo di studio;
  \item \textbf{R}: è la lettera con la quale si fa riferimento ad un rischio che potrebbe essere causato dai requisiti;
  \item \textbf{T}: è la lettera con la quale si fa riferimento ad un rischio che potrebbe essere causato dalle tecnologie.
\end{itemize}

\noindent La pericolosità di ogni rischio è stata suddivisa in tre categorie:
\begin{itemize}
  \item \textbf{Pericolosità alta}: provoca uno sforamento significativo sulle ore previste ed è difficilmente risolvibile nel tempo di progetto;
  \item \textbf{Pericolosità media}: provoca uno sforamento di qualche ora ed è risolvibile;
  \item \textbf{Pericolosità bassa}: provoca uno sforamento trascurabile delle ore pianificate ed è facilmente risolvibile.
\end{itemize}

\begin{longtabu}{|X[1,c]|X[2,c]|X[0.8,c,m]|X[0.8,c,m]|}

  \hline

  \textbf{Codice identificativo} & \textbf{Descrizione} & \textbf{Probabilità} & \textbf{Pericolosità} \\

  \hline

  \risk{S} & Difficoltà nella comprensione di un argomento durante il periodo di studio & Media & Media \\

  \hline

  \risk{R} & Comprensione non corretta di un requisito & Bassa & Media \\

  \hline

  \risk{T} & Difficoltà nell'imparare il linguaggio Solidity & Media & Alta \\

  \hline

  \risk{T} & Poca esperienza con le nuove versioni di Java & Bassa & Media \\

  \hline

  \risk{T} & Difficoltà nell'utilizzare un ambiente di \textit{test} per la \textit{blockchain} Ethereum & Media & Media \\

  \hline

  \risk{T} & Difficoltà nell'utilizzare un ambiente di \textit{test} per la \textit{blockchain} Hotmoka & Media & Media \\

  \hline

  \caption{Analisi dei rischi}
\end{longtabu}
