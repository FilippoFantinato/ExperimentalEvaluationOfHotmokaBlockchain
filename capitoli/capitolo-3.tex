% !TEX encoding = UTF-8
% !TEX TS-program = pdflatex
% !TEX root = ../tesi.tex

\chapter{Il progetto: NFTLab}
\label{cap:nftlab}

%%%%%%%%%%%%%%%%%%%%%%%%%%%%%%%%%%%%%%%%%%%%%%%%%%%%%%%%%%%%%%%%%%%%%%%%%%%%%%%%%

\section{Analisi dei rischi}
Breve analisi dei rischi.

\section{Studio preliminare}

\subsection{Cos'è la blockchain}
Spiegazione di cos'è la blockchain, introducendo la blockchain di BitCoin come termine di paragone per quelle interessate nel mio stage.

\subsubsection{Il \emph{wallet}}
Cos'è il wallet, come viene creato, gestito e il ruolo che ha durante le transazioni nella blockchain.

\subsubsection{Il ruolo importante dei \emph{miners}}
Chi sono i miners e il concetto di rewards in base al processo che svolgono, approfondendo i vari tipi di proof-of-X.

\subsubsection{La crittografia in gioco}
Spiegazione dei sistemi crittografici in gioco nella blockchain.

\subsubsection{Composizione di un blocco in BitCoin}
Com'è composto un blocco in BitCoin.

\subsubsection{Come si svolge una transazione e la sua composizione}
Processo intrapreso da una transazione in BitCoin e la sua composizione.

\subsubsection{Le diramazioni in BitCoin}
Essendo la blockchain un sistema distribuito, qui verrà spiegato come bitcoin gestisce le diramazioni che si generano casualmente.

\subsubsection{I possibili attacchi alla blockchain}
Breve introduzione alla sicurezza della blockchain e dei possibili attacchi a cui può essere sottoposta.

\subsection{La blockchain Ethereum}
Introduzione alla blockchain Ethereum spiegando le differenze con BitCoin.

\subsubsection{Il concetto di smart contract}
Cos'è uno smart contract, introduzione al concetto di gas limit e gas price e il linguaggio Solidity.

\subsubsection{Tipi di transazioni}
Approfondimento dei vari tipi di transazioni che si possono compiere in Ethereum.

\subsubsection{Lo stato di Ethereum}
Come viene gestito lo stato di Ethereum e spiegazione della struttura dati merkle patricia tries.

\subsubsection{Composizione di un blocco}
Com'è composto un blocco in Ethereum e differenze rispetto a BitCoin.

% \paragraph{\emph{Merkle Patricia tries}}

\subsection{La blockchain Hotmoka}
Introduzione alla blockchain Hotmoka spiegando le differenze con Ethereum.

\subsubsection{Cos'è Tendermint e il suo utilizzo in Hotmoka}
Spiegazione di cos'è Tendermint e di come viene utilizzato in Hotmoka.

\subsubsection{Lo stato di Hotmoka}
Come viene gestito lo stato in Hotmoka.

\subsubsection{Gli smart contract in Hotmoka}
Come vengono gestiti gli smart contract in Hotmoka e il linguaggio Takamaka.

\subsection{Lo standard ERC721}
Cos'è lo standard ERC721 per la gestione di NFT.

\subsection{Il protocollo IPFS (InterPlanetary File System)}
Cos'è il protocollo IPFS e come si può utilizzare per la gestione di NFT.

%%%%%%%%%%%%%%%%%%%%%%%%%%%%%%%%%%%%%%%%%%%%%%%%%%%%%%%%%%%%%%%%%%%%%%%%%%%%%%%%%

\section{Strumenti e configurazione}

\subsection{Strumenti di gestione del progetto}
Descrizione degli strumenti utilizzati per la gestione del progetto.

\subsection{Documentazione}
Descrizione degli strumenti utilizzati per la scrittura della documentazione e dei vari diagrammi UML.

\subsection{Ambiente di sviluppo}
Descrizione dell'ambiente di sviluppo utilizzato.

\subsection{Tecnologie usate}
Descrizione dei vari linguaggi di programmazione, framework, librerie e strumenti di build automation.

\subsection{Analisi statica del codice}
Descrizione degli strumenti utilizzati per l'automazione dell'analisi statica del codice.

\subsection{Way of working}
Spiegazione di come viene organizzato e gestito il flusso di sviluppo.

%%%%%%%%%%%%%%%%%%%%%%%%%%%%%%%%%%%%%%%%%%%%%%%%%%%%%%%%%%%%%%%%%%%%%%%%%%%%%%%%%

\section{Analisi dei requisiti}

\subsection{Casi d'uso}

\subsubsection{Attori primari}
Elenco e spiegazione degli attori primari.

\subsubsection{Attori secondari}
Elenco e spiegazione degli attori secondari.

\subsubsection{Diagrammi dei casi d'uso}
Tutti i vari diagrammi dei casi d'uso.

% \paragraph{UC1 - Caricamento di un opera in blockchain}

% \paragraph{UC2 - Vendita di un opera}

% \paragraph{UC3 - Ottenimento di un opera a partire dal suo id}

% \paragraph{UC4 - Ottenimento di un opera a partire dal suo hash}

\subsection{Requisiti funzionali}
Tabella dei requisiti funzionali, con relativa spiegazione e fonte.

\subsection{Requisiti di qualità}
Tabella dei requisiti di qualità, con relativa spiegazione e fonte.

\subsection{Requisiti di vincolo}
Tabella dei requisiti di vincolo, con relativa spiegazione e fonte.

%%%%%%%%%%%%%%%%%%%%%%%%%%%%%%%%%%%%%%%%%%%%%%%%%%%%%%%%%%%%%%%%%%%%%%%%%%%%%%%%%

\section{Smart contract per Ethereum}
Spiegazione dello scopo e dei compiti dello smart contract per Ethereum.

\subsection{Progettazione}
Scelte progettuali, best practices impiegate e design pattern che sono stati utilizzati. 

\subsubsection{Architettura}
Presentazione dell'architettura con diagrammi di package, classe e sequenza.

\subsection{Codifica}
Spiegazione di quanto fatto durante il periodo di codifica, approfondendo parti di codice che ritengo importanti.

\subsection{Verifica}
Spiegazione delle librerie attraverso le quali sono stati implementati i test, numero di test scritti e code coverage raggiunto.

%%%%%%%%%%%%%%%%%%%%%%%%%%%%%%%%%%%%%%%%%%%%%%%%%%%%%%%%%%%%%%%%%%%%%%%%%%%%%%%%%

\section{Lo standard ERC721 per Hotmoka}
Spiegazione del perché è risultata necessaria la scrittura dello standard ERC721 per Hotmoka.

\subsection{Progettazione}
Scelte progettuali, best practices impiegate e design pattern che sono stati utilizzati. 

\subsubsection{Architettura}
Presentazione dell'architettura con diagrammi di package, classe e sequenza.

\subsection{Codifica}
Spiegazione di quanto fatto durante il periodo di codifica, approfondendo parti di codice che ritengo importanti.

\subsection{Verifica}
Spiegazione delle librerie attraverso le quali sono stati implementati i test, numero di test scritti e code coverage raggiunto.

%%%%%%%%%%%%%%%%%%%%%%%%%%%%%%%%%%%%%%%%%%%%%%%%%%%%%%%%%%%%%%%%%%%%%%%%%%%%%%%%%

\section{Smart contract per Hotmoka}
Spiegazione dello scopo e dei compiti dello smart contract per Hotmoka.

\subsection{Progettazione}
Scelte progettuali, best practices impiegate e design pattern che sono stati utilizzati. 

\subsubsection{Architettura}
Presentazione dell'architettura con diagrammi di package, classe e sequenza.

\subsection{Codifica}
Spiegazione di quanto fatto durante il periodo di codifica, approfondendo parti di codice che ritengo importanti.

\subsection{Verifica}
Spiegazione delle librerie attraverso le quali sono stati implementati i test, numero di test scritti e code coverage raggiunto.

%%%%%%%%%%%%%%%%%%%%%%%%%%%%%%%%%%%%%%%%%%%%%%%%%%%%%%%%%%%%%%%%%%%%%%%%%%%%%%%%%

\section{Libreria per l'integrazione con gli smart contract}
Spiegazione dello scopo e dei compiti della libreria per l'integrazione con gli smart contract.

\subsection{Progettazione}
Scelte progettuali, best practices impiegate e design pattern che sono stati utilizzati. 

\subsubsection{Architettura}
Presentazione dell'architettura con diagrammi di package, classe e sequenza.

\subsection{Codifica}
Spiegazione di quanto fatto durante il periodo di codifica, approfondendo parti di codice che ritengo importanti.

\subsection{Verifica}
Spiegazione delle librerie attraverso le quali sono stati implementati i test, numero di test scritti e code coverage raggiunto.

%%%%%%%%%%%%%%%%%%%%%%%%%%%%%%%%%%%%%%%%%%%%%%%%%%%%%%%%%%%%%%%%%%%%%%%%%%%%%%%%%

\section{Validazione e collaudo}
Riassunto dei requisiti soddisfatti e risultato del collaudo.

%%%%%%%%%%%%%%%%%%%%%%%%%%%%%%%%%%%%%%%%%%%%%%%%%%%%%%%%%%%%%%%%%%%%%%%%%%%%%%%%%

\section{Resoconto dei prodotti sviluppati}
Quantità di prodotti software che sono stati sviluppati (in termini di linee di codice) e numero di documenti prodotti.

%%%%%%%%%%%%%%%%%%%%%%%%%%%%%%%%%%%%%%%%%%%%%%%%%%%%%%%%%%%%%%%%%%%%%%%%%%%%%%%%%

\section{Integrazione del prodotto in NFTLab}
Come quanto è stato implementato si integra all'interno del prodotto NFTLab.

%%%%%%%%%%%%%%%%%%%%%%%%%%%%%%%%%%%%%%%%%%%%%%%%%%%%%%%%%%%%%%%%%%%%%%%%%%%%%%%%%

\section{Le conclusioni}
Considerazioni conclusive sul confronto tra le due blockchain Hotmoka e Ethereum.
