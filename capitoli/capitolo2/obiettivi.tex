% !TEX encoding = UTF-8
% !TEX TS-program = pdflatex
% !TEX root = ../../tesi.tex

\section{Obiettivi dello stage}
Lo scopo del seguente stage è di sviluppare la parte di creazione e gestione degli NFT, utilizzando due \textit{blockchain}: Ethereum e Hotmoka. In seguito ho dovuto scegliere la migliore e integrarla con il \textit{back-end} di NFTLab. L'obiettivo, perciò, consiste nello sviluppo di uno \textit{smart contract} per la gestione della compravendita di NFT e di una libreria Java che permetta a NFTLab di interagire con lo \textit{smart contract} caricato in \textit{blockchain}. \newline

\noindent I prodotti attesi al termine dello stage sono i seguenti:
\begin{itemize}
  \item un \textbf{documento tecnico} per l'azienda dove vengono illustrati i seguenti punti:
  \begin{itemize}
    \item le differenze nell'implementazione degli \textit{smart contract} per la gestione di NFT seguendo gli \textit{standard} delle due \textit{blockchain} sopra citate;
    \item motivazioni che hanno spinto verso una soluzione piuttosto che verso l'altra;
    \item spiegazione dell'architettura e le parti più complesse che sono state sviluppate.
  \end{itemize}
  \item le \textbf{implementazioni degli \textit{smart contract}} per le \textit{blockchain} Ethereum e Hotmoka;
  \item l'\textbf{implementazione della libreria} per l'interazione del \textit{back-end} con entrambe le \textit{blockchain}.
\end{itemize}

\section{Obiettivi personali}
In seguito ad una discussione avvenuta precedentemente all'inizio dello stage con il mio \textit{tutor} aziendale, l'ingegnere Fabio Pallaro, sono stati definiti vari obiettivi personali che potessero approfondire ogni aspetto del percorso. \\

\noindent Si farà riferimento agli obiettivi secondo la seguente notazione:
\begin{itemize}
  \item \textbf{O}: per gli obiettivi obbligatori, vincolanti in quanto obiettivo primario richiesto dal committente;
  \item \textbf{D}: per gli obiettivi desiderabili, non vincolanti o strettamente necessari, ma dal riconoscibile valore aggiunto;
  \item \textbf{F}: per gli obiettivi facoltativi, rappresentanti valore aggiunto non strettamente competitivo.
\end{itemize}

\noindent La nomenclatura, perciò, sarà formata nel seguente modo:
\begin{center}
  (O|D|F)[0-9]+
\end{center}

\noindent Gli obiettivi personali che ho conseguito sono riassunti di seguito:
\begin{longtabu}{|c|X|}
  \hline

  \textbf{Obiettivo} & \textbf{Descrizione} \\ \hline

  \textbf{Obbligatorio} & \\

  % O01       & Studio della tecnologia \textit{blockchain} \\
  % O02       & Studio della \textit{blockchain} Ethereum \\
  % O03       & Studio della \textit{blockchain} Hotmoka \\ 
  % O04       & Studio del linguaggio Solidity \\ 
  % O05       & Studio degli standard per la gestione degli NFT \\ 
  O01       & Implementazione del 80\% dei contratti per la gestione di NFT previsti \\
  O02       & Implementazione della libreria per la comunicazione di NFTLab con i contratti sviluppati \\

  \hline

  \textbf{Desiderabile} &  \\
  
  D01       & Implementazione del 100\% dei contratti previsti \\

  \hline

  \textbf{Facoltativo} & \\

  F01       & Implementazione dei \textit{test} relativi alle implementazioni realizzate \\

  \hline

  \caption{Obiettivi dello stage}
\end{longtabu}