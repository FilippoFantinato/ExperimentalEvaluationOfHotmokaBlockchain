% !TEX encoding = UTF-8
% !TEX TS-program = pdflatex
% !TEX root = ../../tesi.tex

\section{Scelta dell'azienda}
La scelta di Sync Lab come azienda dove svolgere lo stage è avvenuta durante l'evento annuale, organizzato dall'Università degli Studi di Padova in associazione con gli imprenditori di AssIndustria VenetoCentro, chiamato \textbf{StageIT}. Questo evento, svoltosi online a causa della pandemia, permette alle aziende di presentare il loro ambito lavorativo e i vari progetti di stage proposti. Durante questo evento ho potuto incontrare varie aziende, analizzando di ognuna le opportunità che offrivano. In seguito ho contattato e sono stato contattato da varie aziende, le quali avevano il mio \textit{curriculum vitae} caricato in precedenza sulla piattaforma SIAGAS offerta dall'Università di Padova. Tra le varie aziende che mi hanno contattato c'è stata anche Sync Lab. \\

Il motivo principale che mi ha spinto a sceglierla come azienda dove svolgere lo stage è \textbf{il progetto}. Infatti, quasi tutti i progetti di stage che mi hanno proposto le altre aziende erano molto pratici, dove bisognava utilizzare un \textit{framework} per sviluppare siti web o applicazioni \textit{mobile}. Il progetto proposto da Sync Lab si discosta molto da quest'ottica, riuscendo ad unire perfettamente innovazione, aspetti teorici e aspetti pratici. Mi ha permesso di approfondire, sotto gli aspetti elencati in precedenza, la tecnologia \textit{blockchain} la quale ho sempre visto con molta curiosità, cercando di capire se è davvero solo una tendenza oppure se consiste in una tecnologia che rivoluzionerà molti ambiti professionali. \\

Un altro motivo per cui ho preferito Sync Lab è stata \textbf{l'azienda in senso stretto}, intesa come ambiente lavorativo, poiché al suo interno è presente un numero elevato di dipendenti ed è particolarmente strutturata. L'ho scelta con la speranza di poter vivere un ambiente lavorativo con grande esperienza dove potermi confrontare con persone molto più esperte di me sotto l'aspetto tecnico e tecnologico.