
%**************************************************************
% Acronimi
%**************************************************************
\renewcommand{\acronymname}{Acronimi e abbreviazioni}

\newacronym[description={\glslink{DApp}{Decentralized application}}]
    {dApp}{DApp}{Decentralized application}

%**************************************************************
% Glossario
%**************************************************************
%\renewcommand{\glossaryname}{Glossario}

\newglossaryentry{criptovalute}{
    name={Criptovalute},
    text={criptovalute},
    sort={criptovalute},
    description={Per criptovaluta si intende una valuta virtuale che costituisce una rappresentazione digitale di valore ed e' utilizzata come mezzo di scambio o detenuta a scopo di investimento. Le criptovalute possono essere trasferite, conservate o negoziate elettronicamente}
}

\newglossaryentry{DApp}{
    name={\glslink{dApp}{DApp}},
    description={Per DApp si intende un'applicazione che lavora in un ambiente distribuito. Le DApp a cui si fa riferimento in questo testo sono quelle basate sulla \textit{blockchain}, programmabili attraverso \textit{smart contract}. Un qualsiasi software, per poter essere definito DApp, deve soddisfare i seguenti criteri: deve essere un'applicazione completamente \textit{open-source}, i vari dati delle operazioni devono essere storicizzati su una \textit{blockchain} pubblica e deve utilizzare un token crittografico generato da lei stessa}
}
