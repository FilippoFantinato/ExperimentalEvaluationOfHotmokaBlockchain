%**************************************************************
% file contenente le impostazioni della tesi
%**************************************************************

%**************************************************************
% Frontespizio
%**************************************************************

% Autore
\newcommand{\myName}{Filippo Fantinato}                                    
\newcommand{\myTitle}{Valutazione sperimentale di Hotmoka Blockchain}

% Tipo di tesi                   
\newcommand{\myDegree}{Tesi di laurea}

% Università             
\newcommand{\myUni}{Università degli Studi di Padova}

% Facoltà       
\newcommand{\myFaculty}{Corso di Laurea in Informatica}

% Dipartimento
\newcommand{\myDepartment}{Dipartimento di Matematica "Tullio Levi-Civita"}

% Titolo del relatore
\newcommand{\profTitle}{Prof.}

% Relatore
\newcommand{\myProf}{Tullio Vardanega}

% Luogo
\newcommand{\myLocation}{Padova}

% Anno accademico
\newcommand{\myAA}{2020-2021}

% Data discussione
\newcommand{\myTime}{Luglio 2021}


%**************************************************************
% Impostazioni di impaginazione
% see: http://wwwcdf.pd.infn.it/AppuntiLinux/a2547.htm
%**************************************************************

\setlength{\parindent}{14pt}   % larghezza rientro della prima riga
\setlength{\parskip}{0pt}   % distanza tra i paragrafi


%**************************************************************
% Impostazioni di biblatex
%**************************************************************
\bibliography{bibliografia} % database di biblatex 

\defbibheading{bibliography} {
    \cleardoublepage
    \phantomsection 
    \addcontentsline{toc}{chapter}{\bibname}
    \chapter*{\bibname\markboth{\bibname}{\bibname}}
}

\setlength\bibitemsep{1.5\itemsep} % spazio tra entry

\DeclareBibliographyCategory{opere}
\DeclareBibliographyCategory{web}

\addtocategory{opere}{womak:lean-thinking}
\addtocategory{web}{site:agile-manifesto}

\defbibheading{opere}{\section*{Riferimenti bibliografici}}
\defbibheading{web}{\section*{Siti Web consultati}}


%**************************************************************
% Impostazioni di caption
%**************************************************************
\captionsetup{
    tableposition=top,
    figureposition=bottom,
    font=small,
    format=hang,
    labelfont=bf
}

%**************************************************************
% Impostazioni di glossaries
%**************************************************************

%**************************************************************
% Acronimi
%**************************************************************
\renewcommand{\acronymname}{Acronimi e abbreviazioni}

\newacronym[description={\glslink{ICT}{Information and Communications Technology}}]
    {ict}{ICT}{Information and Communications Technology}

\newacronym[description={\glslink{apig}{Application Program Interface}}]
    {api}{API}{Application Program Interface}

\newacronym[description={\glslink{umlg}{Unified Modeling Language}}]
    {uml}{UML}{Unified Modeling Language}

%**************************************************************
% Glossario
%**************************************************************
\renewcommand{\glossaryname}{Glossario}
% \renewcommand{\glossar}{Glossario}

\newglossaryentry{ICT}{
    name=\glslink{ict}{ICT},
    text=Information and Communications Technology,
    description={Con il termine \textit{Information and Communications Technology}, si intende l'uso della tecnologia nella gestione e nel trattamento delle informazioni. Include tutti gli ambiti professionali che riguardano la progettazione e lo sviluppo tecnico della comunicazione digitale}
}

\newglossaryentry{System Integrator}
{
    name=System Integrator,
    description={Con il termine inglese \textit{System Integrator} viene indicata un'azienda che si occupa dell'integrazione di sistemi. Il suo compito è quello di far dialogare impianti diversi tra di loro, allo scopo di creare una nuova struttura funzionale che possa utilizzare sinergicamente le potenzialità degli impianti d'origine e creando quindi funzionalità originariamente non presenti}
}

\newglossaryentry{on premises}{
    name={On premises},
    text={on premises},
    description={In informatica con il termine \textit{on premises} si indica l'installazione del \textit{software} direttamente sulla macchina locale}
}

\newglossaryentry{Single Page Application}{
    name=Single Page Application,
    text=Single Page Application,
    description={Per \textit{Single Page Application} si intende qualsiasi applicazione web che interagisce con l'utente aggiornando le parti della pagina con dati prelevati direttamente dal server}
}

\newglossaryentry{Single File Components}{
    name=Single File Components,
    text=Single File Components,
    description={È la principale caratteristica del \textit{framework} Vue.js e permette di definire il codice di \textit{markup}, la logica e lo stile direttamente all'interno dello stesso file}
}
 % database di termini
\makeglossaries


%**************************************************************
% Impostazioni di graphicx
%**************************************************************
\graphicspath{{immagini/}} % cartella dove sono riposte le immagini


%**************************************************************
% Impostazioni di hyperref
%**************************************************************
\hypersetup{
    %hyperfootnotes=false,
    %pdfpagelabels,
    %draft,	% = elimina tutti i link (utile per stampe in bianco e nero)
    colorlinks=true,
    linktocpage=true,
    pdfstartpage=1,
    pdfstartview=,
    % decommenta la riga seguente per avere link in nero (per esempio per la stampa in bianco e nero)
    %colorlinks=false, linktocpage=false, pdfborder={0 0 0}, pdfstartpage=1, pdfstartview=FitV,
    breaklinks=true,
    pdfpagemode=UseNone,
    pageanchor=true,
    pdfpagemode=UseOutlines,
    plainpages=false,
    bookmarksnumbered,
    bookmarksopen=true,
    bookmarksopenlevel=1,
    hypertexnames=true,
    pdfhighlight=/O,
    %nesting=true,
    %frenchlinks,
    urlcolor=webbrown,
    linkcolor=RoyalBlue,
    citecolor=webgreen,
    %pagecolor=RoyalBlue,
    %urlcolor=Black, linkcolor=Black, citecolor=Black, %pagecolor=Black,
    pdftitle={\myTitle},
    pdfauthor={\textcopyright\ \myName, \myUni, \myFaculty},
    pdfsubject={},
    pdfkeywords={},
    pdfcreator={pdfLaTeX},
    pdfproducer={LaTeX}
}

%**************************************************************
% Impostazioni di itemize
%**************************************************************
% \renewcommand{\labelitemi}{$\ast$}

\renewcommand{\labelitemi}{$\bullet$}
%\renewcommand{\labelitemii}{$\cdot$}
%\renewcommand{\labelitemiii}{$\diamond$}
%\renewcommand{\labelitemiv}{$\ast$}


%**************************************************************
% Impostazioni di listings
%**************************************************************
\lstset{
    language=[LaTeX]Tex,%C++,
    keywordstyle=\color{RoyalBlue}, %\bfseries,
    basicstyle=\small\ttfamily,
    %identifierstyle=\color{NavyBlue},
    commentstyle=\color{Green}\ttfamily,
    stringstyle=\rmfamily,
    numbers=none, %left,%
    numberstyle=\scriptsize, %\tiny
    stepnumber=5,
    numbersep=8pt,
    showstringspaces=false,
    breaklines=true,
    frameround=ftff,
    frame=single
} 

\newcommand{\cmark}{\ding{51}}%
\newcommand{\xmark}{\ding{55}}%


%**************************************************************
% Impostazioni di xcolor
%**************************************************************
\definecolor{webgreen}{rgb}{0,.5,0}
\definecolor{webbrown}{rgb}{.6,0,0}

%**************************************************************
% Impostazioni tabu
%**************************************************************
\tabulinesep=1.2mm

%**************************************************************
% Altro
%**************************************************************

\newcommand{\omissis}{[\dots\negthinspace]} % produce [...]

% eccezioni all'algoritmo di sillabazione
\hyphenation
{
    ma-cro-istru-zio-ne
    gi-ral-din
}

\newcommand{\sectionname}{sezione}
\addto\captionsitalian{\renewcommand{\figurename}{Figura}
                       \renewcommand{\tablename}{Tabella}}

\newcommand{\glsfirstoccur}{\textbf{\textsubscript{(g)}}}

\newcommand{\intro}[1]{\textit{\textsf{#1}}}

%**************************************************************
% Environment per ``rischi''
%**************************************************************
\newcounter{riskcounter}                % define a counter
\setcounter{riskcounter}{0}             % set the counter to some initial value

%%%% Parameters
% #1: Title
% \newenvironment{risk}[1]{
%     \refstepcounter{riskcounter}        % increment counter
%     \par \noindent                      % start new paragraph
%     \textbf{\arabic{riskcounter}. #1}   % display the title before the 
%                                         % content of the environment is displayed 
% }{
%     \par\medskip
% }

% \newcommand{\riskname}{Rischio}

% \newcommand{\riskdescription}[1]{\textbf{\\Descrizione:} #1.}

% \newcommand{\risksolution}[1]{\textbf{\\Soluzione:} #1.}

\renewcommand{\arraystretch}{1.2}

%**************************************************************
% Environment per ``use case''
%**************************************************************
\newcounter{usecasecounter}             % define a counter
\setcounter{usecasecounter}{0}          % set the counter to some initial value

%%%% Parameters
% #1: ID
% #2: Nome
\newenvironment{usecase}[2]{
    \renewcommand{\theusecasecounter}{\usecasename #1}  % this is where the display of 
                                                        % the counter is overwritten/modified
    \refstepcounter{usecasecounter}             % increment counter
    \vspace{10pt}
    \par \noindent                              % start new paragraph
    {\large \textbf{\usecasename #1: #2}}       % display the title before the 
                                                % content of the environment is displayed 
    \medskip
}{
    \medskip
}

\newcommand{\usecasename}{UC}

%**************************************************************
% Environment per ``namespace description''
%**************************************************************

\newenvironment{namespacedesc}{
    \vspace{10pt}
    \par \noindent                              % start new paragraph
    \begin{description} 
}{
    \end{description}
    \medskip
}

\newcommand{\classdesc}[2]{\item[\textbf{#1:}] #2}

\newcounter{personalGoalOCounter}
\newcounter{personalGoalDCounter}
\newcounter{personalGoalFCounter}

\setcounter{personalGoalOCounter}{0}
\setcounter{personalGoalDCounter}{0}
\setcounter{personalGoalFCounter}{0}

\newcommand{\personalGoal}[1]{
	\ifthenelse{\equal{#1}{O}}{
        \refstepcounter{personalGoalOCounter}
        \personalGoalCode{#1}{\thepersonalGoalOCounter}
    }
	{\ifthenelse{\equal{#1}{D}}{
        \refstepcounter{personalGoalDCounter}
        \personalGoalCode{#1}{\thepersonalGoalDCounter}
    }
	{\ifthenelse{\equal{#1}{F}}{
        \refstepcounter{personalGoalFCounter}
        \personalGoalCode{#1}{\thepersonalGoalFCounter}
    }{}}}
}

\newcommand{\personalGoalCode}[2]{
    #10#2
}

\newcounter{UCcounter}

\setcounter{UCcounter}{0}

\newcommand{\UC}[1]{
	\stepcounter{UCcounter}
	
	\subsubsection{UC\theUCcounter{} - #1}
	
	\addtocounter{UCcounter}{-1}
	\refstepcounter{UCcounter}
}

\newcommand{\actualUC}{UC\theUCcounter}

\newcommand{\UCPrimaryActors}[1]{\textbf{Attori primari}: #1}
\newcommand{\UCSecondaryActors}[1]{\textbf{Attori secondari}: #1}
\newcommand{\UCPre}[1]{\textbf{Precondizione}: #1}
\newcommand{\UCPost}[1]{\textbf{Postcondizione}: #1}
\newcommand{\usecasedesc}[1]{\textbf{Descrizione}: #1}
\newcommand{\UCMain}[1]{\textbf{Scenario principale}: #1}
\newcommand{\UCAlt}[1]{\textbf{Scenario alternativo}: #1}
\newcommand{\UCExt}[1]{\textbf{Estensioni}: #1}

\newcommand{\lett}{\textbf{\alph*}.}

\tabulinesep=1.2mm

% Requirements Counters

% Fun
\newcounter{rfCounter}
\newcounter{rfoCounter}
\newcounter{rfdCounter}
\newcounter{rfzCounter}

\setcounter{rfCounter}{0}
\setcounter{rfoCounter}{0}
\setcounter{rfdCounter}{0}
\setcounter{rfzCounter}{0}

\newcommand{\rfun}[2]{
	\refstepcounter{rfCounter}
	\rfuncode{#1}{\therfCounter}{#2}

	\ifthenelse{\equal{#1}{O}}{\stepcounter{rfoCounter}}
	{\ifthenelse{\equal{#1}{D}}{\stepcounter{rfdCounter}}
	{\ifthenelse{\equal{#1}{Z}}{\stepcounter{rfzCounter}}}}
}

\newcommand{\rfuncode}[3]{
	\textbf{RF#1#2}\ifthenelse{\equal{#3}{}}{}{\_#3}
}

\newcommand{\actualrfun}{\therfCounter}
\newcommand{\actualrfunO}{\therfoCounter}
\newcommand{\actualrfunD}{\therfdCounter}
\newcommand{\actualrfunZ}{\therfzCounter}

\newcounter{tracfunCounter}

\setcounter{tracfunCounter}{0}

\newcommand{\tracfun}[2]{
	\stepcounter{tracfunCounter}
	\rfuncode{#1}{\thetracfunCounter}{#2}
}

% Qua
\newcounter{rqCounter}
\newcounter{rqoCounter}
\newcounter{rqdCounter}
\newcounter{rqzCounter}

\setcounter{rqCounter}{0}
\setcounter{rqoCounter}{0}
\setcounter{rqdCounter}{0}
\setcounter{rqzCounter}{0}

\newcommand{\rqua}[1]{
	\refstepcounter{rqCounter}
	\rquacode{#1}{\therqCounter}

	\ifthenelse{\equal{#1}{O}}{\stepcounter{rqoCounter}}
	{\ifthenelse{\equal{#1}{D}}{\stepcounter{rqdCounter}}
	{\ifthenelse{\equal{#1}{Z}}{\stepcounter{rqzCounter}}}}
}

\newcommand{\rquacode}[2]{
	\textbf{RQ#1#2}
}

\newcommand{\actualrqua}{\therqCounter}
\newcommand{\actualrquaO}{\therqoCounter}
\newcommand{\actualrquaD}{\therqdCounter}
\newcommand{\actualrquaZ}{\therqzCounter}

\newcounter{tracquaCounter}

\setcounter{tracquaCounter}{0}

\newcommand{\tracqua}[1]{
	\stepcounter{tracquaCounter}
	\rquacode{#1}{\thetracquaCounter}
}

% Vin
\newcounter{rvCounter}
\newcounter{rvoCounter}
\newcounter{rvdCounter}
\newcounter{rvzCounter}

\setcounter{rvCounter}{0}
\setcounter{rvoCounter}{0}
\setcounter{rvdCounter}{0}
\setcounter{rvzCounter}{0}

\newcommand{\rvin}[1]{
	\refstepcounter{rvCounter}
	\rvincode{#1}{\thervCounter}

	\ifthenelse{\equal{#1}{O}}{\stepcounter{rvoCounter}}
	{\ifthenelse{\equal{#1}{D}}{\stepcounter{rvdCounter}}
	{\ifthenelse{\equal{#1}{Z}}{\stepcounter{rvzCounter}}}}
}

\newcommand{\rvincode}[2]{
	\textbf{RV#1#2}
}

\newcommand{\actualrvin}{\thervCounter}
\newcommand{\actualrvinO}{\thervoCounter}
\newcommand{\actualrvinD}{\thervdCounter}
\newcommand{\actualrvinZ}{\thervzCounter}

\newcounter{tracvinCounter}

\setcounter{tracvinCounter}{0}

\newcommand{\tracvin}[1]{
	\stepcounter{tracvinCounter}
	\rvincode{#1}{\thetracvinCounter}
}

% Risk
\newcounter{riskCounter}

\setcounter{riskCounter}{0}

\newcommand{\risk}[1]{
	\refstepcounter{riskCounter}
	\riskcode{#1}{\theriskCounter}
}

\newcommand{\riskcode}[2]{
	\textbf{RI#1#2}
}
